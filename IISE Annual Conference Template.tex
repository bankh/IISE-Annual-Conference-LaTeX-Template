\documentclass[10pt]{article}

%%%%%%%%%%%%%%%%%%%%%%%%%%%%%%%%%%%%%%%%%%%%%%%%%%%%%%%%%%%%%%%%%%%%%%
% IISE Annual Conference Proceedings Template
% Based on official IISE Annual Conference formatting guidelines
% Date: 2025-11-14
% Version: 1.0
% License: CC BY 4.0
% Repository: https://github.com/bankh/IISE-Annual-Conference-LaTex-Template
% Author: Sinan Bank
% Email: sinan.bank@colostate.edu
% Description: This template is based on the official IISE Annual Conference formatting guidelines.
% It is designed to help authors prepare their papers for submission to the IISE Annual Conference.
% It is a professional LaTeX template that is fully compliant with the official IISE Annual Conference formatting guidelines.
% It is a professional LaTeX template that is fully compliant with the official IISE Annual Conference formatting guidelines.
%%%%%%%%%%%%%%%%%%%%%%%%%%%%%%%%%%%%%%%%%%%%%%%%%%%%%%%%%%%%%%%%%%%%%%

%%%%% Page Setup %%%%%%%%%%%%%%%%%%%%%%%%%%%%%%%%%%%%%%%%%%%%%%%%%%%%%
\usepackage[letterpaper, margin=1in, headheight=25pt]{geometry}
\usepackage{times}           % Times New Roman font
\usepackage{graphicx}        % For figures
\usepackage{amsmath}         % For equations
\usepackage{cite}            % For IEEE style citations
\usepackage{url}             % For URLs in references
\usepackage{fancyhdr}        % For headers
\usepackage{titlesec}        % For section formatting
\usepackage{caption}         % For caption formatting
\usepackage{parskip}         % For paragraph spacing
\usepackage{enumitem}        % For bullet formatting
\usepackage{etoolbox}        % For patching environments

%%%%% Paragraph and Spacing Settings %%%%%%%%%%%%%%%%%%%%%%%%%%%%%%%%%
\setlength{\parindent}{0pt}          % No paragraph indentation
\setlength{\parskip}{10pt}           % Single blank line between paragraphs
\linespread{1.0}                     % Single line spacing
\pagestyle{fancy}
\fancyhf{}                           % Clear all headers and footers
\renewcommand{\headrulewidth}{0pt}  % No header line
\renewcommand{\footrulewidth}{0pt}  % No footer line

%%%%% First Page Header (Conference Proceedings Editors' Information) %%%%%
\fancypagestyle{firstpage}{%
    \fancyhf{}
    \fancyhead[L]{{\fontsize{10}{12}\selectfont 
    Proceedings of the 2026 IISE Annual Conference \\
    H. Xiao, H.E. Romeijn, Z. Shen, M. Zhou, eds.}}
    \renewcommand{\headrulewidth}{0pt}
}

%%%%% Section Formatting %%%%%%%%%%%%%%%%%%%%%%%%%%%%%%%%%%%%%%%%%%%%
% Main sections: Numbered, 12pt, bold, left-justified, one blank line above only
\titleformat{\section}
    {\normalfont\fontsize{12}{14}\bfseries\selectfont}
    {\thesection.}{0.5em}{}
\titlespacing*{\section}{0pt}{10pt}{0pt}

% Subsections: Numbered, 10pt, bold, left-justified
\titleformat{\subsection}
    {\normalfont\fontsize{10}{12}\bfseries\selectfont}
    {\thesubsection}{0.5em}{}
\titlespacing*{\subsection}{0pt}{10pt}{0pt}

% Subsubsections: Numbered, 10pt, bold, left-justified
\titleformat{\subsubsection}
    {\normalfont\fontsize{10}{12}\bfseries\selectfont}
    {\thesubsubsection}{0.5em}{}
\titlespacing*{\subsubsection}{0pt}{10pt}{0pt}

%%%%% Caption Formatting %%%%%%%%%%%%%%%%%%%%%%%%%%%%%%%%%%%%%%%%%%%%%
\captionsetup[table]{
    position=top,
    justification=centering,
    singlelinecheck=true,
    labelsep=colon,
    font={normalsize},
    skip=10pt
}

\captionsetup[figure]{
    position=bottom,
    justification=centering,
    singlelinecheck=true,
    labelsep=colon,
    font={normalsize},
    skip=10pt
}

%%%%% Bullet List Formatting %%%%%%%%%%%%%%%%%%%%%%%%%%%%%%%%%%%%%%%%%
\setlist[itemize,1]{label=\textbullet, leftmargin=0.5in}
\setlist[itemize,2]{label=\textopenbullet, leftmargin=0.75in}
\setlist[itemize,3]{label=\textendash, leftmargin=1in}

%%%%%%%%%%%%%%%%%%%%%%%%%%%%%%%%%%%%%%%%%%%%%%%%%%%%%%%%%%%%%%%%%%%%%%

%%%%% Blind Review Flag %%%%%%%%%%%%%%%%%%%%%%%%%%%%%%%%%%%%%%%%%%%%%%
% Set to 0 for non-blind (with author info), 1 for blind review
\def\blind{1}
%%%%%%%%%%%%%%%%%%%%%%%%%%%%%%%%%%%%%%%%%%%%%%%%%%%%%%%%%%%%%%%%%%%%%%

\begin{document}
\thispagestyle{firstpage}

%%%%% Title and Author Information %%%%%%%%%%%%%%%%%%%%%%%%%%%%%%%%%%%

\if0\blind
% NON-BLIND VERSION: Show title, Abstract ID, and author information
\begin{center}
{\fontsize{16}{19}\bfseries\selectfont 
Paper Title Goes Here: Use 16-Point, Times New Roman, Bold, Centered Font}

\vspace{10pt}

{\fontsize{12}{14}\bfseries\selectfont Abstract ID: 1234}

\vspace{10pt}

% Author information (shown in non-blind version)
{\fontsize{10}{12}\selectfont
Author Name$^{a}$ and Co-Author Name$^{b}$ \\
\vspace{5pt}
$^{a}$Department Name, University Name, City, Country \\
$^{b}$Department Name, University Name, City, Country}
\end{center}
\fi

\if1\blind
% BLIND VERSION: Show only title and Abstract ID, no authors
\begin{center}
{\fontsize{16}{19}\bfseries\selectfont 
Paper Title Goes Here: Use 16-Point, Times New Roman, Bold, Centered Font}

\vspace{10pt}

{\fontsize{12}{14}\bfseries\selectfont Abstract ID: 1234}

\vspace{10pt}

% Placeholder space for author information (to maintain consistent spacing)
{\fontsize{10}{12}\selectfont Author information is purposely removed for blind review}
\end{center}
\fi

\vspace{10pt}

%%%%% Abstract Section %%%%%%%%%%%%%%%%%%%%%%%%%%%%%%%%%%%%%%%%%%%%%%%
\begin{center}
{\fontsize{12}{14}\bfseries\selectfont Abstract}
\end{center}

{\fontsize{10}{12}\selectfont
For the abstract section heading, use 12-point, Times New Roman, bold, centered font, followed by a single blank line. Then include the abstract for your paper (max 250 words), using 10-point, Times New Roman, full-justified font. The abstract should provide a concise summary of the paper's objectives, methods, results, and conclusions. Avoid using abbreviations, footnotes, or references in the abstract. Write in the third person and use past tense for completed work. The abstract is a standalone summary that allows readers to quickly understand the key contributions of your work.
}

%%%%% Keywords (12pt, bold, left-justified, NOT numbered) %%%%%%%%%%%%
{\fontsize{12}{14}\bfseries\selectfont Keywords}\newline
{\fontsize{10}{12}\selectfont
Industrial engineering, operations research, stochastic processes
}

%%%%%%%%%%%%%%%%%%%%%%%%%%%%%%%%%%%%%%%%%%%%%%%%%%%%%%%%%%%%%%%%%%%%%%
% MAIN BODY OF PAPER (10pt, Times New Roman, full-justified)
%%%%%%%%%%%%%%%%%%%%%%%%%%%%%%%%%%%%%%%%%%%%%%%%%%%%%%%%%%%%%%%%%%%%%%
% NOTE: Include three to five keywords. This Keywords section does not have a section number.

\section{Page Layout}

All papers are limited to 6 pages (including title and author data, references, and figures and tables) and should follow the following format and layout. Use the \verb|\cite{}| command to reference sources \cite{chang2006}:

Page Size: 8.5" x 11" (Letter). Top and bottom margins: 1.00". Left and right margins: 1.00". First page header: Conference proceedings editors' information (follow this template). Header (from pages two to six): No header for initial submission. The last names of the authors will be included in the header for accepted manuscripts during final submission.

Spacing and Justification: Single line-spacing with full-justified text. Paragraphs: No indentation of paragraphs. Use a single blank line to separate paragraphs. No footers or page numbers.

\section{Text Sections and Headings}

Text should be organized into sections and subsections. A single line should separate paragraphs; no paragraph indentations should be used.

Font guidelines are as follows: Section Headings: Numbered, 12-point, Times New Roman, bold, upper and lower case, left-justified. Leave one blank line above only. Section Sub-headings: Numbered, 10-point, Times New Roman, bold, upper and lower case, left-justified. Regular text: 10-point, Times New Roman, full-justified, with a single line between paragraphs.

\subsection{Section Sub-headings}

This is an example of a subsection with 10-point, Times New Roman, bold font. Leave one blank line between subsections.

\subsubsection{Sub-subsection Example}

This demonstrates a third-level heading, also 10-point, Times New Roman, bold.

\subsection{Bullets}

Bullet guidelines are as follows:

\begin{itemize}
    \item First level bullet
    \begin{itemize}
        \item Second level bullet
        \begin{itemize}
            \item Third level bullet
        \end{itemize}
    \end{itemize}
\end{itemize}

\section{Figures, Tables, and Captions}

Tables and figures should be included in the main text (see Figure~\ref{fig:example} and Table~\ref{tab:example}), as close to the point of their introduction as possible. Figure and table numbering should be independent.

Table captions: 10-point, Times New Roman, centered (if a single line) or full-justified (if multiple lines). Place caption above the table; leave one blank line above and below the caption. For an example, see Table~\ref{tab:example} below.

Figure captions: 10-point, Times New Roman, centered (if a single line) or full-justified (if multiple lines). Place caption below the figure; leave one blank line above and below. As an example, see Figure~\ref{fig:example} below.

\begin{table}[h!]
\centering
\caption{Example table listing preventive maintenance parameters for an electric distribution case study}
\begin{tabular}{|l|c|c|c|}
\hline
\textbf{Critical component (X$_n$)} & $\alpha_n$ & $\beta_n$ & $C_n^{PM}$ \\
\hline
Transmission System 2 & 2.0 & 1.5 & 7.5 \\
Distribution Company 1 & 1.2 & 1.75 & 8.0 \\
Distribution Company 2 & 1.0 & 1.75 & 12.0 \\
Distribution Company 3 & 1.3 & 1.75 & 10.5 \\
\hline
\end{tabular}
\label{tab:example}
\end{table}

This is an example paragraph to demonstrate the guidelines for the table (above) and figure (below).

\begin{figure}[h!]
\centering
% Replace 'example-figure.pdf' with your actual figure file
% Supported formats: .pdf, .png, .jpg
\fbox{\parbox{0.6\textwidth}{\centering\vspace{1in}[Your Figure Here]\vspace{1in}}}
% \includegraphics[width=0.6\textwidth]{example-figure.pdf}
\caption{Example figure demonstrating (a, b) the concept of ultrasound-assisted 3D-biofabrication process, and (c, d) experimental results with aligned cells. A figure could include a single or multiple schematics, line and CAD drawings, photos, graphs etc. arranged in a concise layout. Labels within the figure should be legible.}
\label{fig:example}
\end{figure}

\section{Equations}

Equations should be centered and numbered in order of appearance. Place the equation number in parentheses, positioned flush to the right margin. Prepare equations with an Equation Writer; screenshots of equations should not be used. Leave one line before and after every equation. See Equation~(\ref{eq:example}) below as an example.

\begin{equation}
t_{align} = \frac{9\lambda\eta\rho_a}{2\pi kr^2 \Phi} \ln\left(\frac{\tan(kx_0)}{\tan(kx_f)}\right)
\label{eq:example}
\end{equation}

As shown in Equation~(\ref{eq:example}), mathematical expressions should be properly formatted.

%%%%%%%%%%%%%%%%%%%%%%%%%%%%%%%%%%%%%%%%%%%%%%%%%%%%%%%%%%%%%%%%%%%%%%
% ACKNOWLEDGEMENTS (No section number)
%%%%%%%%%%%%%%%%%%%%%%%%%%%%%%%%%%%%%%%%%%%%%%%%%%%%%%%%%%%%%%%%%%%%%%

\section*{Acknowledgements}

\if0\blind
% NON-BLIND VERSION: Include acknowledgements with specific details
Acknowledgement of funding support and/or any other kind of assistance should be contained in an ``Acknowledgements'' section (this section should have no section number), located immediately before the ``References and Citations'' section.
\fi

\if1\blind
% BLIND VERSION: Anonymized acknowledgements
Acknowledgements are purposely removed for blind review to maintain anonymity.
\fi

%%%%%%%%%%%%%%%%%%%%%%%%%%%%%%%%%%%%%%%%%%%%%%%%%%%%%%%%%%%%%%%%%%%%%%
% REFERENCES (IEEE Style, No section number)
%%%%%%%%%%%%%%%%%%%%%%%%%%%%%%%%%%%%%%%%%%%%%%%%%%%%%%%%%%%%%%%%%%%%%%

\section*{References and Citations}

The IEEE style \cite{ieee2018} should be followed for references and citations for each manuscript submitted to the IISE Annual Conference. In brief, references are numbered sequentially by order of occurrence (not alphabetically) in the text and listed in a separate section labeled References (no section number) at the end of the manuscript. Within the text, they should be cited by the corresponding list number in square brackets. When referring to two references, use this format \cite{yodo2016,tan2012}. If you refer to more than two documents listed consecutively, use this format \cite{ieee2018,yodo2016,tan2012,chang2006}. The following provides example formats for different types of reference documents.

\vspace{10pt}

%%%%% OPTION 1: Inline Bibliography (Manual) %%%%%%%%%%%%%%%%%%%%%%%%%
% Uncomment the section below if you prefer inline references:

% \begin{thebibliography}{9}
% \setlength{\itemsep}{0pt}
% \setlength{\parskip}{0pt}
% \bibitem{ieee2018}
% Institute of Electrical and Electronics Engineers, ``IEEE Reference Guide,'' IEEE Periodicals, V 11.12.2018, 2018. [Online]. Available: https://ieee-dataport.org/IEEECitationGuidelines.pdf. [Accessed September 16, 2020].
% \bibitem{yodo2016}
% N. Yodo, P. Wang, ``Resilience Allocation for Early Stage Design of Complex Engineered Systems,'' \textit{Journal of Mechanical Design}, vol. 138, no. 9, pp. 091402 (10 pages), July 2016, doi: 10.1115/1.4033990.
% \bibitem{tan2012}
% Z. Tan, R. A. Shirwaiker, ``A Review of the State of Art and Emerging Industrial and Systems Engineering Trends in Biomanufacturing,'' in \textit{Proceedings of 2012 Industrial and Systems Engineering Research Conference}, Orlando, FL, USA, May 19-23, 2012, pp. 2407-2414.
% \bibitem{chang2006}
% T. Chang, R. Wysk, H. Wang, \textit{Computer-Aided Manufacturing}, 3rd ed. Upper Saddle River, NJ, USA: Prentice Hall, 2006.
% \end{thebibliography}

%%%%% OPTION 2: BibTeX (Recommended) %%%%%%%%%%%%%%%%%%%%%%%%%%%%%%%%%
% Using BibTeX for reference management (edit references.bib file)
\bibliographystyle{ieeetr}

% Remove spacing between bibliography items to match IISE format
\AtBeginEnvironment{thebibliography}{%
  \setlength{\itemsep}{0pt}%
  \setlength{\parskip}{0pt}%
  \setlength{\parsep}{0pt}%
}

\bibliography{references}

\end{document}

